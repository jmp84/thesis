\chapter{Refinements in System Development for Machine Translation}
\chaptermark{System Development in MT}

% TODO check bib through online tool

In this chapter, we present language model and grammar refinements
techniques that give improvements in translation. These techniques
may be used in order to build the best possible system for a
translation evaluation or to customize a translation system
for a client in the industry setting.

Experiments are based on a system submitted to the WMT13
Russian-English translation shared
task~\citep{pino-waite-xiao-degispert-flego-byrne:2013:WMT}.
We briefly summarize the system building in
\autoref{sec:wmt13ExperimentalSetup}.
In \autoref{sec:domainAdaptationMT}, we review domain adaptation
techniques for machine translation. In \autoref{sec:domainAdaptationLM},
we show how to adapt a language model to obtain better performance
on a specific domain such as newswire text.
In \autoref{sec:domainAdaptationGrammar}, we show how additional
features related to specific domains may help in translation.
Finally, in \autoref{sec:bestPossibleRescoring} and \autoref{sec:sbVSkn}, we investigate
the best possible strategy for combining first pass translation
and language model rescoring in terms of language model
training data size, $n$-gram language model order and
language model smoothing.

\section{Experimental Setup}
\label{sec:wmt13ExperimentalSetup}

The experiments reported in this chapter are based
on the system submitted to the WMT13 Russian-English translation shared
task~\citep{pino-waite-xiao-degispert-flego-byrne:2013:WMT}.
In this section, we summarize the system building.

We use all the Russian-English parallel data available in the constraint track.
We filter out non Russian-English sentence pairs with the
\emph{language-detection} library.\footnote{http://code.google.com/p/language-detection/}
A sentence pair is filtered out if the language detector detects a different language with probability
more than 0.999995 in either the source or the target.
This discards 78543 sentence pairs. In addition, sentence pairs where the source sentence has no Russian character, defined by the
Perl regular expression [{\textbackslash}x{0400}-{\textbackslash}x{04ff}], are discarded.
This further discards 19000 sentence pairs.

The Russian side of the parallel corpus is tokenised with the
Stanford CoreNLP toolkit.\footnote{http://nlp.stanford.edu/software/corenlp.shtml}
The English side of the parallel corpus is tokenised with a standard English tokeniser.
Both sides of the parallel corpus are then lowercased, so mixed case is restored in post-processing.
Corpus statistics after filtering are summarised in
\autoref{tab:parallelStatsWMT13}.
%
\begin{table}[htbp]
\begin{center}
\begin{tabular}{*{3}{|r}|}
\hline
Lang & \# Tokens & \# Types \\
\hline
\hline
%Russian & CoreNLP & 2445919 & 47426938 & 1191325 \\
RU & 47.4M & 1.2M \\
\hline
%English & Aachen & 2445919 & 50419263 & 711001 \\
EN & 50.4M & 0.7M \\
\hline
\end{tabular}
\end{center}
\caption{Russian-English parallel corpus statistics.}
\label{tab:parallelStatsWMT13}
\end{table}
%
Parallel data is aligned using the MTTK
toolkit~\citep{deng-and-byrne:2008:ASLP}.
We train a word-to-phrase HMM model with a maximum phrase length of 4 in both
source-to-target and target-to-source directions. The final alignments are obtained
by taking the union of alignments obtained in both directions.
A synchronous context-free grammar~\citep{chiang:2007:CL}
is extracted from the alignments. The constraints
are set as in the original publication with the following exceptions:
%
\begin{itemize}
  \item phrase-based rule maximum number of source words: 9
  \item maximum number of source element (terminal or nonterminal): 5
  \item maximum span for nonterminals: 10
\end{itemize}
%
Maximum likelihood estimates for the translation probabilities are computed using
MapReduce as described in \autoref{chap:hfile}.
We are using shallow-$1$ hierarchical grammars~\cite{degispert-iglesias-blackwood-banga-byrne:2010:CL} in our 
experiments. This model is constrained enough that the decoder can build exact search spaces,
i.e. there is no pruning in search that may lead to spurious undergeneration errors.

For first-pass decoding, we use HiFST as described in \autoref{sec:hifst}. First-pass
decoding is followed by a large 5-gram language model rescoring step
as described in \autoref{sec:rescoring}.

\section{Domain Adaptation for Machine Translation}
\label{sec:domainAdaptationMT}

Let us first formalize the problem of domain adapatation.
In a standard multi-class classification problem, we are
given a training set
$\{(x_i, y_i) \in \mathcal{X} \times \mathcal{Y}, i \in [1, N]\}$
where $\mathcal{X}$ is a set of instances to be labelled and
$\mathcal{Y}$ is a finite set of labels. Machine translation
can be seen as a multi-class classification problem where $\mathcal{X}$
is the set of source sentences and $\mathcal{Y}$ is the set of
target sentences. The multi-class classification learning problem
is to find a function $f : \mathcal{X} \rightarrow \mathcal{Y}$
that minimizes the number of prediction errors.

A \emph{domain} is a particular distribution $\mathcal{D}$
over $\mathcal{X}$. For example, $\mathcal{D}$ can be such that
source sentences in $\mathcal{X}$ drawn from $\mathcal{D}$ are in the newswire
domain. For domain adaptation, we assume a out-of-domain distribution
$\mathcal{D}_O$ and a in-domain distribution $\mathcal{D}_I$.
A model is trained on
a sample drawn from $\mathcal{D}_O$ but the model performance
is evaluated on a sample drawn from $\mathcal{D}_I$.
\emph{Domain adaptation} consists in altering the training procedure
by using information from domain
$\mathcal{D}_I$ in order to achieve better performance on $\mathcal{D}_I$.
The out-of-domain and in-domain distributions are also called
source domain and target domain respectively.

In machine translation, we can distinguish two types of domain
adaptation problem: \emph{cross-domain} adaptation and \emph{dynamic}
adaptation~\citep{foster-kuhn:2007:WMT}. In cross-domain adaptation,
a small amount of in-domain training data is available while
in dynamic adaptation, no in-domain training data is available.
Dynamic adaptation may be important for online translation systems for example.
Our concern is cross-domain adaptation, where at least an in-domain
development set is available for parameter tuning.

%notes on Experiments in Domain Adaptation for Statistical Machine Translation by Keohn and Schroeder
%large europarl // data, small newscommentary // data, test set in news domain
%baseline: concatenate the parallel corpora
%in domain data to train language model
%interpolated language model: train 2 lms: one in out of domain, one in in domain
%grid search on interpolation weights
%use the 2 lms as separate features
\citet{koehn-schroeder:2007:WMT} explore various techniques for
machine translation domain adaptation. They use the
Europarl~\citep{koehn:2005:MTSummit} as out-of-domain
training data and a news commentary parallel
corpus\footnote{\url{http://statmt.org/wmt07/shared-task.html}}
for in-domain training data. Training the language model
on in-domain training data gives an improvement of 0.8 BLEU
with respect to the baseline. Training two separated language models
on the in-domain and out-of-domain training data and interpolating
them with weights set to minimize perplexity on the tuning set
gives an improvement of 0.4 BLEU. Using these two language models
as separate features to be optimized by MERT~\citep{och:2003:ACL}
gives an improvement of 0.6 BLEU. Finally, using two separate
translation models trained on the in-domain and out-of-domain
data and tuning the weights with MERT gives an improvement
of 0.9 BLEU. These results are reported on the tuning set only.
In this chapter, we explore very similar techniques.

%notes on Domain Adaptation for Statistical Machine Translation with Monolingual Resources by Bertoldi and Federico
%exploit monolingual in domain data (src or trg)
%adapt sp-en system from UN to Europarl
%distinguish cross-domain and dynamic adaptation
%large in-domain monolingual data
%use directly to adapt lm
%generate synthetic // corpus and adapt translation and reordering model
%generate synthetic // data directly with alignment given by decoder
%phrase table is the union and the translation features are smoothed by some
%lexical prob in case one phrase is not extracted from one of the // corpora

% TODO add this
%\citet{bertoldi-federico:2009:WMT} show how to exploit
%a monolingual corpus for machine translation domain
%adaptation. An in-domain monolingual corpus is available
%either on the source or target side. A source side
%monolingual corpus is automatically translated with
%an out-of-domain baseline system. The two resulting
%phrase tables are merged and the translation features
%for phrase pairs not extracted from one of the corpus
%are smoothed with a lexical feature. The language model
%is trained on the target side of the synthetic corpus.
%This strategy gives a gain of 1 BLEU. Using in-domain
%monolingual data on the target side is much more effective.
%Simply training the language model on this data gives
%a gain of 5.2 BLEU. Creating a synthetic parallel corpus
%and merging the phrase tables as just described gives
%an additional 0.3 BLEU.

% notes on Discriminative Instance Weighting for Domain Adaptation
% in Statistical Machine Translation by foster goutte kuhn
%baseline adaptation technique
%obvious: use mert tune set in domain
%more difficult: how to adapt lm and s2t and t2s
%no adaptation of alignment
%concatenate in and out training data
%train two models and interpolate. one way to do this
%is two use them as separate features in mert. drawback is
%that mert becomes unstable
%log linear combination not good because multiplies prob (?)
%use linear intepolation instead
%optimization on in domain dev set for LM
%for TM not so straightforward
%hat{alpha} = argmax_alpha sum_s,t ptilda(s,t) log sum_i alpha_i p_i(s | t)
%ptilda(s,t): joint empirical distrib on in domain dev set using normal
%rule extraction
%alternative: use MAP (see paper for formula)
%sentence selection: filter out-of-domain training to match
%input source sentences
%sentence selection crude. matsoukas et al use sentence pair weighting.
%extension: learn weights on phrase pairs
%model
%p(s|t) = alpha_t p_I(s | t) + (1 - alpha_t) p_o(s | t)
%p_I: s2t as usual computed on the in domain corpus
%p_o: instance weighted model computed on the out of domain corpus



%papers to review:
%eck et al 2004
%hildebrand et al 2005 (similar to eck et al 2004)
%foster and kuhn 2007
%finch sumita 2008
%civera and juan 2007
%ueffing et al 2007
%schwenk 2008
% daume 2007, daume marcu 2007
% matsoukas et al 2009

\section{Language Model Adaptation for Machine Translation}
\label{sec:domainAdaptationLM}

In this section, we contrast two simple language model adaptations
techniques for machine translation. Experiments are carried out
on the Russian-English translation shared task for
WMT13\footnote{\url{http://statmt.org/wmt13/translation-task.html}}.
The translation system is described in a separate
publication~\citep{pino-waite-xiao-degispert-flego-byrne:2013:WMT}
and summarised in \autoref{sec:wmt13ExperimentalSetup}.
In this section, contrary to the system submitted to the WMT13
shared task, we do not use any provenance features.

We used the KenLM toolkit~\citep{heafield-pouzyrevsky-clark-koehn:2013:ACL} to
estimate separate 4-gram LMs with modified Kneser-Ney
smoothing~\citep{kneser-ney:1995:ICASSP,chen-goodman:1998:harvard}, for each of the
corpora listed in \autoref{tab:monolingualStats}. The component models were then
interpolated with the SRILM toolkit~\citep{stolcke:2002:SLP} to form a single LM.
The interpolation weights were optimised for perplexity on the \emph{news-test2008},
\emph{newstest2009} and \emph{newssyscomb2009} development sets, using
the \emph{compute-best-mix} tool, which is part of the SRILM toolkit.
The weights reflect
both the size of the component models and the genre of the corpus the component models
are trained on, e.g. weights are larger for larger corpora in the news genre.
We also train a modified Kneser-Ney smoothed 4-gram LM using
all concatenated data from \autoref{tab:monolingualStats}.
%
\begin{table}[htbp]
\begin{center}
\begin{tabular}{|l|r|}
\hline
Corpus & \# Tokens \\
\hline
EU + NC + UN + CzEng + Yx & 652.5M \\
Giga + CC + Wiki & 654.1M \\
News Crawl & 1594.3M \\
afp & 874.1M \\
apw & 1429.3M \\
cna + wpb & 66.4M \\
ltw & 326.5M \\
nyt & 1744.3M \\
xin & 425.3M \\
\hline
Total & 7766.9M \\
\hline
\end{tabular}
\end{center}
\caption{Statistics for English monolingual corpora.}
\label{tab:monolingualStats}
\end{table}  
%
\autoref{tab:lmInterpolationBestStrategy} shows that the best strategy for
building a language model is to do offline linear interpolation to minimize
perplexity on a development set that has the same genre as the translation
development sets. The second best strategy is to use an uninterpolated
language model. The worst strategy is to do a log-linear interpolation
of the various language model components by tuning these language
language models as separate features with lattice MERT. Note that
these observations are confirmed even after rescoring with a large
stupid-backoff 5-gram language model.
One possible reason for the log-linear interpolation of various
language models performs the worst might be that the MERT algorithm
becomes unstable with more features~\citep{foster-kuhn:2009:WMT}.
%
\begin{table}
  \begin{center}
    \begin{tabular}{l|lll}
      Configuration & newstest2012.tune & newstest2012.test & newstest2013 \\
      \hline
      Uninterpolated LM & 33.22 & 32.01 & 25.35 \\
      +5g & 33.33 & 32.26 & 25.53 \\
      \hline
      Interpolated LM & 33.71 & 32.26 & 25.47 \\
      +5g & 33.74 & 32.51 & 25.58 \\
      \hline
      Several Components & 33.23 & 31.75 & 25.37 \\
      +5g & 33.28 & 31.85 & 25.48
    \end{tabular}
    \caption{Performance comparison between an uninterpolated language model, a
    linearly interpolated language model and a log-linearly interpolated language mode.
    Off-line linear interpolation is the best strategy.}
    \label{tab:lmInterpolationBestStrategy}
  \end{center}
\end{table}

% TODO do exp that compares linear and loglinear interp of ONLY 2 lms.

\section{Domain Adaptation with Provenance Features}
\label{sec:domainAdaptationGrammar}

% notes on chiang's paper
%run porter stemmer on en side of // corpus
%build two word translation table
%t(e' | f ) and t(e|e')
%t_m(e | f ) = sum_{e'} t(e' | f) t(e | e')
%t_m(f | e) = sum_{e'} t(f | e') t(e' | e) = t(f | e')
%conditioning on provenance
%each sentence pair has genre/collection info
%compute word translation tables t_s(e | f) and t_s(f | e) for each feature s
%for unseen word pairs, use Witten-Bell smoothing
%for each s, for phrase pair/rule (f,e) add two features - log t_s(e | f)/t(e|f)

In the previous section, we have presented domain adaptation strategies
for the language model. In this section, we will focus on domain
adaptation for the translation model.

\citet{chiang-deneefe-pust:2011:ACL} introduce \emph{provenance} lexical features.
The lexical feature $t(\bm{e} \mid \bm{f})$ for a phrase
pair $(\bm{f}, \bm{e})$ is defined in \autoref{eq:lexicalFeature}:
%
\begin{equation}
  t(\bm{e} \mid \bm{f}) = \prod_{i = 1}^{|\bm{e}|}
  \begin{cases}
    \frac{1}{|a_i|} \sum_{j \in a_i} t(e_i \mid f_j) & \text{if } |a_i| > 0 \\
    t(e_i \mid \text{NULL}) & \text{otherwise}
  \end{cases}
  \label{eq:lexicalFeature}
\end{equation}
%
where $t(e_i \mid f_j)$ is a word translation table computed from the
word aligned parallel text. Each sentence pair is marked with
its \emph{provenance}, for example what genre, such as newswire
or web, it belongs to. Word translation tables and the
lexical feature are computed for each provenance.

We use a very similar approach with the following differences and
extensions:
%
\begin{itemize}
  \item The lexical features is computed differently, as described in
    \autoref{sec:features}.
  \item The features added to the system are not the ratio between the
    provenance specific lexical feature and the general lexical feature but
    simply the provenance specific lexical feature.
  \item We extend this technique to compute provenance specific
    source-to-target and target-to-source translation models
    as described in \autoref{sec:rulextractMapReduce}.
\end{itemize}
%
For our experiments, we use 4 provenances that correspond
to the 4 subcorpora provided: the Common Crawl corpus, the
News Commentary corpus, the Yandex corpus and the Wiki Headlines
corpus. This gives an additional 16 features to our system.
%
\begin{table}
  \begin{center}
    \begin{tabular}{l|lll}
      Configuration & newstest2012.tune & newstest2012.test & newstest2013 \\
      \hline
      No Provenance & 33.71 & 32.26 & 25.47 \\
      +5g           & 33.74 & 32.51 & 25.58 \\
      \hline
      Provenance & 33.22 & 32.14 & 25.40 \\
      +5g        & 33.22 & 32.14 & 25.41 \\
      \hline
      Provenance Union & 33.65 & 32.36 & 25.55 \\
      +5g              & 33.67 & 32.58 & 25.63 \\
%      \hline
%      Provenance Union No Provenance & 33.22 & 31.78 & 25.59 \\
%s      +5g                            & 33.22 & 31.78 & 25.59 \\
    \end{tabular}
    \caption{Comparison between the baseline and the use of
    provenance feature, with or without the union strategy.}
    \label{tab:noprovVsProvVsProvunion}
  \end{center}
\end{table}
%
Comparing the first two rows of \autoref{tab:noprovVsProvVsProvunion},
we can see that in our setting provenance features are not
useful. However, if we use the provenance union strategy described
in \autoref{sec:rulextractMapReduce}, we can see that having additional
rules together with provenance features is the best strategy.
In order to verify whether the gains are only due to additional
rules, we rerun the experiment in row 3 and remove the provenance
features. After rescoring, we obtain 33.22 BLEU on newstest2012.tune,
31.78 BLEU on newstest2012.test and 25.59 BLEU on newstest2013, which
demonstrates the usefulness of provenance features when using the
provenance union strategy.

\section{Training Data Size}
\label{sec:bestPossibleRescoring}

In this section, we give recommendations on how much
data should be used to train a first pass language model
and what order should be chosen for the first pass
$n$-gram language model.

Our systems are typically run in two steps. The
first step usually consists in decoding with
a 4-gram language model. The second step consists
in 5-gram lattice rescoring.
\citet{brants-popat-xu-och-dean:2007:EMNLP-CoNLL} argue
that single pass decoding is conceptually simpler
and may lose less information. We attempt to verify this
by incorporating 5-gram language models directly in
first-pass decoding.

Comparing the first two rows of \autoref{tab:lmSizes}, we
can see that more language training data for the first
pass language model helps both in first-pass decoding
and in rescoring. Comparing the last two rows, we can
see that more language training data is helpful only
at the first pass decoding step. Comparing the first and
third rows, and the second and fourth rows, we can see
that for equal amounts of training data, the best strategy
is to use a first-pass 4-gram language model followed by
large 5-gram language model rescoring. One caveat with this
conclusion of course is that the maximum amount of data experimented
with is 7.8B words as opposed to 2 trillion words reported
by \citet{brants-popat-xu-och-dean:2007:EMNLP-CoNLL}.
%
\begin{table}
  \begin{center}
    \begin{tabular}{l|lll}
      Configuration & newstest2012.tune & newstest2012.test & newstest2013 \\
      \hline
      medium 4g & 32.96 & 31.53 & 24.60 \\
      +5g &       33.43 & 32.12 & 25.30 \\
      \hline
      large 4g  & 33.71 & 32.26 & 25.47 \\
      +5g       & 33.74 & 32.51 & 25.58 \\
      \hline
      medium 5g & 32.62 & 31.62 & 24.77 \\
      +5g       & 33.20 & 31.96 & 25.27 \\
      \hline
      large 5g  & 32.99 & 31.79 & 25.18 \\
      +5g       & 32.99 & 31.79 & 25.18 \\
    \end{tabular}
    \caption{Comparing various data size conditions for language model training
    and different $n$-gram language model orders. In conclusion, more language
    model training data is helpful but the use of higher order language
    model is only beneficial in rescoring.}
    \label{tab:lmSizes}
  \end{center}
\end{table}

% TODO make the large 5g first pass interpolated by using restricted vocab.

\section{Smoothing for 5-gram Rescoring}
\label{sec:sbVSkn}

In this section, we contrast the use of a stupid-backoff 5-gram
language model and a modified Kneser-Ney smoothed language model
for lattice rescoring. We train a stupid-backoff 5-gram language model
and a modified Kneser-Ney 5-gram language model on all available
English data for WMT13, described in \autoref{tab:monolingualStats}.

On average, over the eight experiments from \autoref{tab:SB5gVsKN5g}
we obtain a gain of 0.10 BLEU on the tune set, 0.12 BLEU on the first
test set and 0.11 BLEU on the second test set. We conclude that the
use of modified Kneser-Ney smoothing is slightly beneficial in 5-gram
lattice rescoring.

\begin{table}
  \begin{center}
    \begin{tabular}{l|lll}
      Configuration         & newstest2012.tune & newstest2012.test & newstest2013 \\
      \hline
      Interpolated large 4g & 33.71 & 32.26 & 25.47 \\
      + SB 5g               & 33.74 & 32.51 & 25.58 \\
      + KN 5g               & 33.73 & 32.26 & 25.48 \\
      \hline
      Uninterpolated large 4g & 33.22 & 32.01 & 25.35 \\
      + SB 5g                 & 33.33 & 32.26 & 25.53 \\
      + KN 5g                 & 33.31 & 32.28 & 25.65 \\
      \hline
      Several Components      & 33.23 & 31.75 & 25.37 \\
      + SB 5g                 & 33.28 & 31.85 & 25.48 \\
      + KN 5g                 & 33.30 & 31.92 & 25.35 \\
      \hline
      Provenance              & 33.22 & 32.14 & 25.40 \\
      + SB 5g                 & 33.22 & 32.14 & 25.41 \\
      + KN 5g                 & 33.54 & 32.37 & 25.75 \\
      \hline
      Provenance Union        & 33.65 & 32.36 & 25.55 \\
      + SB 5g                 & 33.67 & 32.58 & 25.63 \\
      + KN 5g                 & 33.89 & 32.82 & 25.84 \\
      \hline
      Interpolated medium 5g  & 32.62 & 31.62 & 24.77 \\
      + SB 5g                 & 33.20 & 31.96 & 25.27 \\
      + KN 5g                 & 33.24 & 32.29 & 25.47 \\
      \hline
      Interpolated medium 4g  & 32.96 & 31.53 & 24.60 \\
      + SB 5g                 & 33.43 & 32.12 & 25.30 \\
      + KN 5g                 & 33.65 & 32.46 & 25.55 \\
      \hline
      Uninterpolated large 5g & 32.99 & 31.79 & 25.18 \\
      + SB 5g                 & 32.99 & 31.79 & 25.18 \\
      + KN 5g                 & 32.99 & 31.79 & 25.18 \\
    \end{tabular}
    \caption{Comparison between the gains obtained by stupid-backoff
    5-gram rescoring and Kneser-Ney 5-gram rescoring.}
    \label{tab:SB5gVsKN5g}
  \end{center}
\end{table}

\section{Conclusion}

In this chapter, we have investigated various refinements
on translation system building. We come to the following
conclusions for building higher quality systems:
%
\begin{itemize}
  \item In order to build a language model for cross-domain
    adaptation, the best strategy is to build an interpolated
    language model and tune the interpolation weights in order
    to minimize the perplexity on a development set in the
    target domain, as opposed to tuning the language model
    weights with MERT.
  \item For systems that employ a two pass procedure, using
    a first pass 4-gram language model performs better than
    using a first pass 5-gram language model.
  \item Finally, even with large amounts of monolingual training
    data at the order several billion words, language model smoothing
    strategies are important in the lattice rescoring setting.
\end{itemize}

