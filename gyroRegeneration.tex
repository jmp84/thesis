\chapter{String Regeneration as Phrase-Based Translation}
\chaptermark{String Regeneration as Phrase-Based MT}
\label{chap:gyro}

% TODOFINAL canonical word count: refer to background
% TODOFINAL grep for <s> and </s> and use texttt
% TODOFINAL British vs US spelling
% TODOFINAL grep config grep -v configuration and replace
% TODOFINAL do a kitchen sink exp with all goodies (either with chop or not)
% TODOFINAL grep for mentions of BLEU without case
% TODOFINAL check if mentioned that columns are sorted by score
% TODOFINAL point to github for thesis source
% TODOFINAL maybe say somewhere that unigrams not found in the n-grams are added
% TODOFINAL score intermediate system for NIST 12 (see email Gonzalo for the references)
% TODOFINAL grep for natural language generation and replace by nlg
% TODOFINAL finish sig tests for wmt chapter
% TODOFINAL maybe do sig tests for this chapter
% TODOFINAL (done) grep for sec:optimization and replace with sec:evaluationMetrics where appropriate
% TODOFINAL background: lmbr applied to get the oracle + grep for oracle
% TODOFINAL (already done) future work idea: more sophisticated future cost language model for gyro trans
% TODOFINAL normalize ngram n-gram $n$-gram etc.
% TODOFINAL (done already ?) shorter name for the chapter
% TODOFINAL (done already ?) make subsubsection numbered
% TODOFINAL (done already ?) discuss look ahead

In the previous chapters, we have proposed various improvements to hierarchical
phrase-based translation systems, in terms of
infrastructure (\autoref{chap:hfile}), grammar
modelling (\autoref{chap:extractionFromPosteriors} and \autoref{chap:wmt})
and language modelling for first pass decoding as well as
rescoring (\autoref{chap:wmt}). In the
chapter to follow (\autoref{chap:gyroTrans}), we will also
introduce techniques for potential improvements in fluency in the output
of hierarchical phrase-based systems. In this chapter, we lay
the ground work to be applied in translation in \autoref{chap:gyroTrans}.

Machine translation output was originally evaluated by
human judges in terms
of \emph{adequacy}, or how well the meaning of the source
text is preserved in the translated text, and \emph{fluency},
or how well-formed the translation is~\citep{white-oconnell-carlson:1993:HLT},
i.e.\ to what extent it is syntactically, morphologically, and orthographically
correct~\citep{reiter-dale:1997:JNLE}.
Adequacy and fluency motivate the
separation of SMT models
into a translation model and a language model (see \autoref{sec:sourceChannelModel}).
There is increasing concern that translation output
often lacks fluency~\citep{knight:2007:TALK}.
In this chapter, we study the related problem of
string regeneration: given a sentence where the
word order has been scrambled, how difficult is it
to recover the original sentence ? Our approach
is inspired by phrase-based translation techniques and
achieves state-of-the-art performance on the string regeneration
task. Software written for this chapter is available
online.\footnote{\url{https://github.com/jmp84/NgramGen}}

% TODOFINAL end of abstract too quick

\section{Introduction}

Before the widespread use of automatic metrics such as
BLEU (see \autoref{sec:evaluationMetrics}) or
METEOR~\citep{banerjee-lavie:2005:MTSumm}, machine translation
systems were originally evaluated in terms of adequacy
and fluency.
There is increasing concern that translation output
often lacks fluency~\citep{knight:2007:TALK}.
Current popular automatic metrics for translation can be ``gamed'' in the
sense that it is possible to obtain a high score according to these
metrics with an output that lacks fluency.
For example, a translation may obtain a high BLEU score simply
because it contains many $n$-grams that are also in the reference
translation but this translation may not be well-formed syntactically.
\autoref{fig:gamingBLEUexample} shows how an improvement of
20 BLEU points at the sentence level between an MT hypothesis
and an oracle MT hypothesis, i.e. the best performing hypothesis among a set, may
not translate into
an improvement in fluency. Of course, BLEU is designed to
measure quality at the set level, so hopefully other oracle MT
hypotheses will probably improve fluency.
The nature of the translation metrics may have led research on translation to neglect fluency.
%
\begin{figure}
\begin{quote}
  \textbf{SMT System (19.76 Sentence-Level BLEU)}: enda according to jerzy made a televised speech, the president of burundi civil war of resistance army soldiers in the past ten years, and had to settle on january 5 before the completion of the new military leadership will be in place before january 7, four of the former rebel army leader.

  \textbf{SMT System Oracle (39.91 Sentence-Level BLEU)}: en, a us \$according to al made a televised speech, said that the president of the burundi's rebel army soldiers in the past 10-year civil war must be before january 5 and the completion of the new military leadership will be in place by january 7, and four of the former leaders of the rebel forces.

  \textbf{Reference}: According to President Ndayizeye's televised speech, former rebel fighters in Burundi's 10-year civil war must be settled in by January 5.  The new army leadership is to be made up of 40 percent former rebels and should be in place by January 7.
\end{quote}
\caption{Example showing how a large improvement in BLEU score does not imply an improvement in fluency.
  The first quote is one of the output sentences produced by an SMT system presented in \autoref{chap:gyroTrans}.
  The second quote is the oracle translation for that system, i.e.\ the best possible translation to be found
  in the lattice of translations generated by the system. The third quote is one of the human reference translation.}
\label{fig:gamingBLEUexample}
\end{figure}

Even if one strives to produce a fluent translation, it remains a very difficult task.
Different word ordering between distant language pairs such as
Chinese-English is one of the main reasons for the lack of fluency in translation.
In general, obtaining a correct word order is a fundamental problem in many natural language
processing applications.
In machine translation, because
source language and target language have a different word
ordering, a machine translation system needs to model word
reordering. In some language pairs such as French-English, the
source language gives good cues to the target word order, however,
in other language pairs such as Chinese-English or Japanese-English,
the word order can be very different in the source and
the target language. In natural language generation tasks, such
as paraphrasing or summarisation, word ordering
is also critical at the surface realisation
step~\citep{reiter-dale:1997:JNLE}.
Word ordering is also one possible type of grammatical
error, especially for foreign language
learners~\citep{yu-chen:2012:COLING}. % TODOFINAL do something with this last crazy sentence

In this chapter, we study the word ordering problem in
isolation through the task of
string regeneration~\citep{wan-dras-dale-paris:2009:EACL}.
Given an input sentence where the word order has
been scrambled, the string regeneration task consists in recovering
the original sentence. This task can be seen as one of the
steps in surface realisation for language generation where
the input is a list of content words and function words and
the output a grammatical sentence. It also allows us to study
the realisation task independently of the translation task.

We use a simple but extensible approach, inspired from
stack-based decoding for phrase-based
translation (see \autoref{sec:phraseBasedDecoding}).
Given a bag of word as input, we use $n$-gram rules
extracted from a large monolingual corpus and concatenate
these $n$-grams in order to form hypotheses. The hypotheses are scored
with a linear model in the same manner as translation
hypotheses in phrase-based translation.
One of the critical features in this system
is the language model. In experiments to follow, we only study
this main language model feature, however, our decoder has other features
implemented and
arbitrary features can be added easily to the
system, demonstrating the system flexibility.

Our strategy achieves state-of-the-art performance as
measured by BLEU (\autoref{sec:evaluationMetrics}) in the string regeneration
task. We also study various capabilities of our decoder: in
\autoref{sec:phraseBasedDecoding}, we show how to
split the input into chunks for more rapid experimentation;
in \autoref{sec:gyroPruning}, we analyse the effect of stack-based pruning;
in \autoref{sec:gyroEffectNgramOrder}, we study
the effect of including $n$-gram rules with a larger $n$-gram order;
in \autoref{sec:overlap}, we relax decoding constraints by allowing
$n$-gram rules to overlap; in \autoref{sec:gyroFutureCost}, we introduce
future cost estimation using a unigram language model; finally, in
\autoref{gyro:rescoring}, we demonstrate how SMT rescoring techniques
are also applicable to the output of our decoder.
% TODOFINAL read gyro paper eacl


% TODONEVER ?
%In addition, we present
%an alternative version that incorporates a dependency language
%model~\citep{shen-xu-weischedel:2008:ACL,shen-xu-weischedel:2010:CL}.

\section{Background}
\label{sec:gyroBackground}

% TODOFINAL probably need to cite these guys: (Bangalore and Rambow, 2000; Langkilde, 2000),
% TODOFINAL describe dependency tree linearization

\citet{brown-cocke-dellapietra-dellapietra-jelinek-lafferty-mercer-roossin:1990:CL}
anecdotally mention the string regeneration task, or \emph{bag of word translation},
in order to demonstrate the effectiveness
of $n$-gram language models. With a 3-gram language model, they recover
63\% of a sentences with less than 11 words with brute force search
over all permutations. This is of course impractical for longer
sentences. % TODOFINAL run WSJ23 with 5g and LMBR and recompute this number
%For a loose comparison, our system is able to recover
%60\% of the tokenized and lowercased sentences with less than 11 words
%on Section 23 of the Penn Treebank.
% TODONEVER sys comb with MT posteriors (same as eacl paper)
% TODOFINAL compare our system to this performance
% TODOFINAL update this performance with a better system or with rescored system
% TODOFINAL maybe: run regeneration with base noun phrase together
% TODOFINAL run gyro with unigrams only for the baseline and try to beat it
% TODOFINAL or TODONEVER manual oracle for word reordering ?

String regeneration has recently gained interest as
a standalone
task.%~\citep{wan-dras-dale-paris:2009:EACL,he-wang-guo-liu:2009:ACLIJCNLP,zhang-clark:2011:EMNLP,zhang-blackwood-clark:2012:EACL2012,zhang:2013:IJCAI}.
%
%wan et al:
%find the best dep tree
%related work: statistical surface realisation Langkilde and Knight
%text to text generation TODOFINAL check for papers with that title
%use string regeneration as surrogate for grammaticality test
%content selection is abstracted out
%training on the ptb
%base noun phrase are kept together !!!!!!!!
%string regeneration: generation component of MT, paraphrase generation, summarisation
%debunk claim that lm performs bad: try to get score with unigrams only
%use chu liu edmond algo: get a dep tree.
%now need to get right ordering for example for all left children
%use language model. this is an example of tree linearization
%
%
\citet{wan-dras-dale-paris:2009:EACL} introduced the string regeneration
task as a proxy to measure the grammaticality of a
natural language generation system output.
String regeneration can also be considered as a simplified version
of the \emph{linguistic realisation} task in a natural language
generation system~\citep{reiter-dale:1997:JNLE}, which consists in
generating a surface form that is syntactically, morphologically,
and orthographically correct.

\citet{wan-dras-dale-paris:2009:EACL} use a
modified version of the minimum spanning tree algorithm
in order to find an optimal dependency tree that covers an input bag of words.
Once a dependency tree is built, an $n$-gram language model is used
so as to order siblings in the tree.
Performance is measured by the BLEU metric.
\citet{wan-dras-dale-paris:2009:EACL} demonstrate BLEU
improvements when regenerating Section 23 of the
Penn Treebank using their original algorithm vs.\ baseline
algorithms based on $n$-gram language models only. Base nouns
indicated by the Penn Treebank are kept together for experimentation.


\citet{he-wang-guo-liu:2009:ACLIJCNLP} also use the dependency tree
formalism for string realisation. They describe a sentence
realiser that takes as input an unordered dependency tree, i.e.\ a dependency
tree that only encode head-modifier relations but not the surface
string ordering. Because an exhaustive search is intractable (due to
the number of possible children permutations at each node), \citet{he-wang-guo-liu:2009:ACLIJCNLP}
adopt a divide-and-conquer approach that recursively linearises
subtrees in a bottom-up fashion. A log-linear model is used
to select the best possible linearisation for each subtree.
Features include dependency relations, $n$-gram language models
for the surface string and for the sequence of heads.

CCG grammars~\citep{zhang-clark:2011:EMNLP} have also been used
for the string regeneration task as an alternative
to the dependency grammar formalism. Given an input bag
of words, the best CCG parse that covers the input if searched
for. Similarly to previous work, input base nouns are kept as
single units. In follow-up work~\citep{zhang-blackwood-clark:2012:EACL2012},
CCG grammars are combined with large language models
and provide performance improvements as measured by the BLEU
metric.

%Dependency-based language model have also been used for the
%string realisation task~\citep{guo-wang-vanGenabith:2011:JNLE}.
%Given an unordered
Some of the works described so far are instances of \emph{tree linearisation}: given
a set of unordered dependency, reconstruct a dependency tree which gives
the word ordering~\citep{belz-bohnet-mille-wanner-white:2012:INLG}.
\citet{zhang:2013:IJCAI} extends this task to partial tree linearisation: the
input is a set of possibly missing POS tags and possibly missing unordered
dependencies. This last work achieves state-of-the art performance on string regeneration
on the Penn Treebank, when using dependency information on the input.
\citet{zhang:2013:IJCAI} also report results on string regeneration without
using dependency information on the input but keeping base nouns as single
units.

In this chapter, we restrict the input to be a bag of words without any additional
syntactic or semantic information.
Our work is directly inspired by recent work that exploits phrase-based
grammar together with a hierarchical phrase-based decoder based on
FSTs (see \autoref{sec:hifst})~\citep{degispert-tomalin-byrne:2014:EACL}. We
also use phrase-based grammars, but
our decoding procedure is analogous to stack-based decoding for phrase-based
translation.
To our knowledge, our method achieves state-of-the-art performance
on this task. We will now present our model, which is analogous
to the phrase-based translation model, but in a monolingual setting.


% TODOFINAL MAYBE THIS
% Towards Developing Generation Algorithms for Text-to-Text Applications
% TODOFINAL (already done ?) also review Adria's paper

\section{Phrase-Based Translation Model for String Regeneration}
\label{sec:gyroPhraseBasedModel}

In the previous section, we have reviewed various strategies for string
regeneration. We observe that, in general, the use of syntax was beneficial
over the exclusive use of an $n$-gram language model. We do not take advantage
of syntactic information, instead, we complement an $n$-gram language
with $n$-gram rules extracted from monolingual data. We now describe our method, which is inspired from
phrase-based translation techniques (see \autoref{sec:phraseBasedTranslation}).
We employ a feature based linear model (see \autoref{sec:loglinearModel}) as our
objective function:
%
\begin{equation}
  \bm{\hat{e}} = \argmax_{\bm{e}} \bm{\lambda} \cdot \bm{h(e)}
  \label{eq:gyroModel}
\end{equation}
%
where:
%
\begin{itemize}
  \item $\bm{\hat{e}}$ is the best possible hypothesis according to the model.
  \item $\bm{e}$ is an English sentence.
  \item $\bm{\lambda}$ is a feature weight vector.
  \item $\bm{h(e)}$ is a feature vector for the hypothesis $\bm{e}$.
\end{itemize}
%
$\bm{\Phi}(\bm{e})$ contains a language model
$g(\bm{e})$ as well as local features $\bm{\varphi(e)}$
that are additive over phrases that segment $\bm{e}$.
Similarly, $\bm{\lambda}$ has a weight $\lambda_g$ for the language model
and weights $\bm{\lambda_{\varphi}}$ for local features. As in phrase-based
translation, a hypothesis
is decomposed into phrases $\bm{e}_1^I$. We can then rewrite
\autoref{eq:gyroModel} into \autoref{eq:gyroModel2}:
%
\begin{equation}
  \begin{split}
    \hat{\bm{e}} &= \argmax_{\bm{e} = \bm{e}_1^I} \lambda_{g} g(\bm{e}) + \bm{\lambda_{\varphi}} \cdot \bm{\varphi(e)} \\
                 &= \argmax_{\bm{e} = \bm{e}_1^I} \lambda_{g} g(\bm{e}) + \bm{\lambda_{\varphi}} \cdot \sum_{i = 1}^I \bm{\varphi(e_i)} \\
  \end{split}
  \label{eq:gyroModel2}
\end{equation}
%
Note that this model makes a max approximation: various segmentations of
the hypotheses are considered but the score only depends on one single
segmentation.
Experiments to follow only make use of the language model feature but
our implementation has other features integrated, such as
word count or rule count, and arbitrary features can be added easily.
%TODONEVER maybe add experiments with additional features
Our decoder algorithm described subsequently in
\autoref{sec:gyroDecoderAlgorithm} generates hypotheses left to right
by concatenating phrases $\bm{e_1}$, ..., $\bm{e_I}$. For a phrase index $i \in [1, I]$,
we define the partial model score $\text{sc}(\bm{e_1^i})$ of partial hypothesis $\bm{e_1^i}$ as
in \autoref{eq:gyroPartialScore}:
%
\begin{equation}
  \text{sc}(\bm{e_1^i}) = \lambda_g g(\bm{e_1^i}) + \bm{\lambda_{\varphi}} \cdot \sum_{k = 1}^i \bm{\varphi(e_k)} \\
  \label{eq:gyroPartialScore}
\end{equation}

\section{Phrase-Based Translation Rules for String Regeneration}
\label{sec:gyroPhraseBasedRules}

In the previous section, we have introduced our model, inspired from
phrase-based translation, for string regeneration.
This model assumes that a hypothesis is segmented into phrases.
In this section, we describe how to obtain these phrases
from monolingual data. In \autoref{sec:gyroRuleExtraction}, we review
the $n$-gram rule extraction process, which is based the Hadoop implementation
of MapReduce. In \autoref{sec:ngramRuleFiltering}, we describe how to obtain
relevant rules for a given test set.

\subsection{Rule Extraction}
\label{sec:gyroRuleExtraction}

$n$-gram rules are extracted from large collections of monolingual
text. The text is tokenised and lowercased. We then
employ the MapReduce
framework as implemented by Hadoop in order to extract % TODOFINAL refer to background
$n$-grams together with their occurrence counts from the corpus.

The input key to the \emph{map} function is irrelevant.
The input value to the \emph{map} function is a line of text.
The \emph{map} function simply extracts all $n$-grams from the
input line of text and outputs these $n$-grams with a count of
one. The input to the \emph{reduce} function is an $n$-gram
and a series of constant values equal to one.
The \emph{reduce} function simply sums the counts
and produces the $n$-gram with its total count. This MapReduce
job is analogous to the canonical ``word count'' example except
that here $n$-grams of higher order than unigrams are considered.
For our experiments, we extract $n$-grams from order 1 to 5. % TODOFINAL : in one instance we extracted 6-grams, made no difference

\subsection{Rule Filtering}
\label{sec:ngramRuleFiltering}

Once we have extracted all $n$-grams from a monolingual corpus, we
would like to retrieve the $n$-grams relevant to a test set
that we wish to experiment on. An $n$-gram is relevant if the vocabulary
of the $n$-gram is included in any of the vocabularies of each
sentence in the test set.

Contrary to the process described in \autoref{sec:rulextract}, we
do not generate queries and then seek those queries in an HFile.
This is because the number of possible queries is too large
to fit in memory. For a sentence of length $N$ with words distinct
from each other, the number of possible relevant 5-grams is
$N \times (N - 1) \times (N - 2) \times (N - 3) \times (N - 4)$.
For sentence of length 100, this would correspond to approximately
9B queries. Even with an alternative implementation where keys in the
HFile are lexicographically sorted $n$-grams and values are the
permutations and counts found in
the monolingual data, the number of 5-gram queries would be $N \choose 5$, which
is approximately 75M queries for a 100 word sentence.

We therefore take the alternative approach of scanning the set of $n$-grams
and retaining the relevant ones. For this, we again use a MapReduce
job. The input key to the \emph{map} function is an $n$-gram and
the input value is a count. For each test sentence, the \emph{map}
function checks if the $n$-gram vocabulary is included in the
vocabulary of that sentence and if so, it generates all possible
coverages of this $n$-gram for that sentence, where a coverage
indicates the positions of the words covered by the $n$-gram
in the input sentence. The \emph{map}
output key is the $n$-gram together with a sentence id. The \emph{map}
output value is a coverage and a count. The \emph{reduce} function
simply removes duplicates in the list of coverages. % TODOFINAL explain why
% TODOFINAL explain coverage.
% TODOFINAL maybe say why we have the counts
% TODOFINAL brief example for the mapreduce job

\section{String Regeneration Search}
\label{sec:gyroDecoderAlgorithm}

In \autoref{sec:gyroPhraseBasedModel}, we have described our
string regeneration model and in \autoref{sec:gyroPhraseBasedRules},
we have described how to extract $n$-gram rules. We now
present our search algorithm and introduce our
string regeneration decoder, \emph{NgramGen}, which
reorders an input bag of word using the model introduced and
stack-based decoding techniques reviewed in \autoref{sec:phraseBasedDecoding}.

\subsection{Input}

The input to the decoder consists of:
%
\begin{itemize}
  \item A bag of words, or multiset, $\bm{w}$, represented as a vector.
    For example, we have $\bm{w} = [\text{do, you, do}]$ for the
    input sentence ``do you do'' to be regenerated.
  \item A set of $n$-grams, where the
    vocabulary of each $n$-gram is a subset of the vocabulary of the
    input bag of words. Each $n$-gram is also associated with
    one or more coverages as computed in \autoref{sec:ngramRuleFiltering}.
    For a given $n$-gram, its coverage is a bit vector
    that indicates what positions the $n$-gram covers in the input bag of words.
    In case of repeated words, an $n$-gram can have several possible
    coverages. For example, with the input bag of words
    $[\text{do, you, do}]$, the $n$-gram $[\text{do, you}]$ has
    two possible coverages: 110 and 011.
    The input $n$-grams and coverages are represented as
    an associative array datastructure. We denote this
    input by $\mathcal{N}$. Note that it would be possible for the decoder
    to compute possible coverages on the fly, however we choose to precompute those
    at the filtering step described in \autoref{sec:ngramRuleFiltering}.
  \item A $n$-gram language model $\mathcal{G}$. The $n$-gram
    language model is used as a the main feature in the model to
    encourage fluent output.
\end{itemize}
%
We will now describe how hypotheses, i.e.\ permutations of the
input, are generated by the decoder from a bag of words.

\subsection{Algorithm}

Our algorithm is inspired from stack based decoding for phrase-based
translation, reviewed in \autoref{sec:phraseBasedDecoding}. We
first give an overview
before outlining the algorithm details.

\subsubsection{Overview}
\label{sec:gyroAlgorithmOverview}

In order to represent partial hypotheses, we use a vector
of \emph{columns} (more commonly called stacks). Each
column contains a set of
\emph{states}. Each state represents a set of partial hypotheses.
All states in a column represent hypotheses that cover
the same number of words in the input.

We start to build hypotheses from an initial state that
represents the empty hypothesis. This initial state
is located in the first column, which represents the
hypotheses that have not covered any word in the input.
The empty hypothesis is then extended with the input
$n$-gram rules and their coverage. New states are created
to take into account these extensions. Then, partial
hypotheses represented by these states are repeatedly
extended until hypotheses that cover the entire input
are produced.

\subsubsection{Algorithm Details}
\label{sec:gyroAlgoDetails}

As mentioned in \autoref{sec:gyroAlgorithmOverview}, we represent
the set of
partial hypotheses with a vector of columns that we denote
$M$. For an input $\bm{w}$ of size $|\bm{w}|$, $M$ has a size
of $|\bm{w}| + 1$. Elements of $M$ are denoted $M_0$, ..., $M_{|\bm{w}|}$.
The $i$-th column $M_i$ represents hypotheses that have covered
$i$ words in the input. The columns $M_i$ are the analog of stacks
in stack based decoding for phrase-based translation.

Each column contains a set of states that represents a set of partial
hypothesis. A state contains the information necessary to
extend a partial hypothesis and represents a set of
partial hypotheses that can
be \emph{recombined} (see \autoref{sec:phraseBasedHypothesisRecombination}).
This means that if two partial hypotheses $h_1$ and $h_2$
are represented by a state $s$ and $h_1$ has a higher model score than
$h_2$, then if $h_1$ and $h_2$ are extended with the same rules, then
the extension of $h_1$ will have a higher score than the extension $h_2$.
In our case, two hypotheses can be recombined if they share the same last
$n - 1$ words where $n$ is the order of the $n$-gram language model.

The states are sorted by their \emph{score}.
For a given state, its score is the best model score of any of the partial
hypotheses that the state represents.
The decoding algorithm is presented in \autoref{alg:gyroDecoder}.
%
\begin{figure}
  \begin{algorithmic}[1]
    \Function{Decode}{$\bm{w}, \mathcal{N}, \mathcal{G}$}
      \State{\Call{Initialize}{$M$}} \hypertarget{alg:line:initM}{} \label{alg:line:initM}
      \For{$0 \leq i < |\bm{w}|$} \hypertarget{alg:line:loopCoverage}{} \label{alg:line:loopCoverage}
        \For{$\text{state } s \in M_i$} \hypertarget{alg:line:loopState}{} \label{alg:line:loopState}
          \For{$\text{ngram } r \in \mathcal{N}$} \hypertarget{alg:line:loopNgram}{} \label{alg:line:loopNgram}
            \If{\Call{CanApply}{$s, r$}} \hypertarget{alg:line:canApply}{} \label{alg:line:canApply}
              \State{\Call{Extend}{$s, r$}} \hypertarget{alg:line:extend}{} \label{alg:line:extend}
            \EndIf
          \EndFor
        \EndFor
      \EndFor
    \EndFunction
  \end{algorithmic}
  \caption{NgramGen decoding algorithm. The input is a bag of words $\bm{w}$,
  a set $\mathcal{N}$ of $n$-gram rules with their coverage and a language model
  $\mathcal{G}$. The column vector $M$ is first initialised with a state representing
  the empty hypothesis. Then, the initial state is iteratively extended.}
  \label{alg:gyroDecoder}
\end{figure}
%
We first initialise the matrix $M$ to be a vector of empty
columns of size $|\bm{w}| + 1$ (\hyperlink{alg:line:initM}{line \ref{alg:line:initM}}).
The first column is filled with an initial state representing an empty hypothesis.
Then, we loop over the column indices
(\hyperlink{alg:line:initM}{line \ref{alg:line:loopCoverage}}), the states in each
column (\hyperlink{alg:line:loopState}{line \ref{alg:line:loopState}}) and the
$n$-gram rules (\hyperlink{alg:line:loopNgram}{line \ref{alg:line:loopNgram}}).
In each case, we attempt to extend a state with an $n$-gram rule. We first test
if the $n$-gram rule $r$ is applicable to state $s$
(\hyperlink{alg:line:canApply}{line \ref{alg:line:canApply}}). If this
is the case, we extend the state $s$ with the rule $r$
(\hyperlink{alg:line:extend}{line \ref{alg:line:extend}}).
We will now describe what kind of information is contained
in a state. This will allow use to also describe the last two operations, \textsc{CanApply}
and \textsc{Extend}.

\paragraph{State Definition}
\label{sec:gyroStateDefinition}

The states must
contain enough information
in order to be able to extend a hypothesis, and
states also represent
all hypotheses that can be recombined. Thus, a state
contains the following information:
%
\begin{itemize}
  \item Coverage: the coverage is a bit vector that indicates
    which words in the input have been covered. This implies a
    sorting of the input bag of words. The sorting is arbitrary
    and we simply choose to represent the bag of words by the
    sequence of words to be recovered. % TODOFINAL put this when defining bag of words ??
  \item History: in order to compute the $n$-gram language model
    score correctly when extending a partial hypothesis, we
    store the last $n-1$ words of the partial hypothesis we want
    to extend. The definition of history is therefore the same
    as the definition of history for an $n$-gram language model.
    In our implementation, the history is simply encoded as
    an \texttt{lm::ngram::State} as defined in the KenLM
    toolkit~\citep{heafield:2011:WMT}.
\end{itemize}

\paragraph{\textsc{CanApply} Operation}
\label{sec:canApply}

Given a state $s$ with coverage $c(s)$ and an $n$-gram rule $r$ with
coverage $c(r)$, rule $r$ can be applied to state $s$ if $c(s)$ and
$c(r)$ are disjoint. We will see in \autoref{sec:overlap}
that this constrain can be relaxed if we allow for overlapping
$n$-grams.

\paragraph{\textsc{Extend} Operation}

Given a state $s$ with coverage $c(s)$ and and $n$-gram rule $r$
with coverage $c(r)$ that can be applied to $s$, we extend
$s$ with $r$ into a new state $s'$ as follows:
%
\begin{itemize}
  \item The coverage $c(s')$ of $s'$ is the bitwise OR
    of $c(s)$ and $c(r)$:
%
\begin{equation}
  c(s') = c(s) \mid c(r)
\end{equation}
%
  \item The new partial score $\text{sc}(s')$ for $s'$ is defined in terms
    of the partial score $\text{sc}(s)$ for $s$, the history $h(s)$ and the rule $r$:
%
\begin{equation}
  \text{sc}(s') = \text{sc}(s) + \lambda_g g(r | h(s)) + \bm{\lambda_{\varphi}} \cdot \bm{\varphi(r)}
\end{equation}
%
  \item If $s'$ already exist in the column corresponding to its coverage, then
    the score of the existing state is updated if $\text{sc}(s')$ is better than
    the existing score. Otherwise, $s'$ is simply added as a new state to the column corresponding
    to its coverage.
\end{itemize}
%

\paragraph{Pruning}
\label{sec:gyroPruningDescription}

The search space for an input bag of size $N$ is the number of permutations
of the input, which is $N!$ if all input words are distinct. We therefore need to apply
pruning during search to make the decoding process tractable.
We support histogram pruning and threshold
pruning (see \autoref{sec:phraseBasedPruning}).
After each extension, if we add a new state, we enforce that the column where
a state was added satisfies the pruning constraints. This means that for
histogram pruning, with $m$ the maximum number of states per column, if we
add a new state to a column that has $m$ states, then the state with the lowest
score is removed from that column.

For threshold pruning with a threshold $t$, the situation is
slightly more complex. When we want to add a state $s$ with score $\text{sc}(s)$ to a column where the best score
is $b$, we consider three cases:
%
\begin{itemize}
  \item $\text{sc}(s) \leq b$ and $\text{sc}(s) \geq b - t$: the state $s$ is added to the column.
  \item $\text{sc}(s) \leq b$ and $\text{sc}(s) < b - t$: the state $s$ is not added to the column.
  \item $\text{sc}(s) > b$: the state $s$ is added to the column and all states in that
    column with a score less than $\text{sc}(s) - t$ are removed from the column.
\end{itemize}

We have presented the decoding algorithm and various data structures to support
the algorithm. We will now describe how the set of hypotheses is encoded with
FSTs.

\subsection{FST Building}
\label{sec:gyroFstBuilding}

We mentioned in \autoref{sec:gyroAlgoDetails} that a state
represents partial hypotheses that can
be recombined as well as the necessary information to be able
to extend a partial hypothesis. However, since we do not keep backpointers
between states,
we cannot recover hypotheses directly from $M$, the vector of columns.
Instead, we build an FST that contains all hypotheses. For each extension,
an arc is added to the output FST. We make use of the OpenFst
toolkit~\citep{allauzen-riley-schalkwyk-skut-mohri:2007:CIAA}.
This allows us to rescore the output using tools already integrated
with OpenFst~\citep{blackwood:2010:PHD}.
Another advantage is that hypotheses that are recombined are
not deleted and all completed hypotheses are stored in the output FST.
This allows hypotheses
that would be discarded in recombination
to remain in the output and get an opportunity to become the best hypothesis
after rescoring.

\subsection{Example}

We now demonstrate how our algorithm works with an example.
Our input consists of:
%
\begin{itemize}
  \item a bag of words: $[\text{do, you, do}]$. Note that the word \emph{do} is repeated.
  \item a bigram language model. Thus the history for each state will consist of the
    last word of the partial hypothesis.
  \item monolingual $n$-gram rules with their coverage of the input bag of word. We
    use the rules listed in \autoref{tab:monolingualRules}.
\end{itemize}
%
\begin{table}
  \begin{center}
  \begin{tabular}{l|l}
    N-gram & Coverage \\
    \hline
    \emph{do} & 100 \\
    \emph{do} & 001 \\
    \emph{you do} & 011 \\
    \emph{you do} & 110 \\
    \emph{do you} & 110 \\
    \emph{do you} & 011 \\
  \end{tabular}
  \caption{Example of input monolingual n-gram rules with their coverage, corresponding
    to the example input $[\text{do, you, do}]$. The FST representing the set of hypotheses
    obtained with these rules is represented in \autoref{fig:exampleAlgoGyro}.}
  \label{tab:monolingualRules}
  \end{center}
\end{table}
%
After running our algorithm without pruning, we obtain the data structure
represented in \autoref{fig:exampleAlgoGyro}.
%
\begin{figure}
  \scriptsize
%  \tikzstyle{State} = [circle, draw, text width = 1.5cm, align = center]
  \tikzstyle{every state}=[text width = 1.5cm, align = center]
  \tikzstyle{line} = [draw, -latex']

  \begin{center}
    \begin{tikzpicture}
      % Place nodes
      \node [state] (init) {000, $<$/s$>$};
      \node [state, above right = 1cm and 1cm of init] (do) {100, do};
      \node [state, right = 2cm of init] (youdo) {011, do};
      \node [state, below right = 1cm and 2.53cm of init] (doyou) {110, you};
      \node [state, right = 2.5cm of do, accepting] (dodoyou) {111, you};
      \node [state, right = 1cm of youdo, accepting] (doyoudo) {111, do};
      % Draw edges
      \path [line] (init) -- node[above, sloped]{do} (do);
      \path [line] (init) -- node[above, sloped]{you do} (youdo);
      \path [line] (init) -- node[above, sloped]{do you} (doyou);
      \path [line] (do) -- node[above, sloped]{do you} (dodoyou);
      \path [line] (do) -- node[above, sloped]{you do} (doyoudo);
      \path [line] (youdo) -- node[above, sloped]{do} (doyoudo);
      \path [line] (doyou) -- node[above, sloped]{do} (doyoudo);
    \end{tikzpicture}
  \end{center}
  \caption{Example of output obtained from running the NgramGen decoder
    on the input $[\text{do, you, do}]$ with the rules listed
    in \autoref{tab:monolingualRules}. Double circled accepting states
    represent completed hypotheses.}
  \label{fig:exampleAlgoGyro}
\end{figure}
%
We can see that the same state \{111, do\} is reused for
the two hypotheses \emph{do you do}, \emph{you do do}, because
these hypotheses have the same coverage and the same history.
We also note that the hypothesis \emph{do you do} is repeated.
This is not an issue as we use an operation of determinisation
on the resulting FST.

We will now present various experiments that demonstrate the effectiveness
of our decoder for the string regeneration task. We will also
analyse various capabilities of the decoder in order to
understand which configurations lead to better results.

\section{Experiments}

In this section, we will present various
string regeneration experiments.  We first present
a baseline without restriction on the input length.
Then, we show how
the input can be split into chunks of maximum length for more rapid experimentation. % TODOFINAL mention that this will be done in translation and mention confidence regions
We then study the effect of pruning, $n$-gram rule
length, the use of overlap between rules, the benefits of future cost
estimation and finally
the benefits of applying rescoring to our first
pass decoder.

\subsection{Experimental Setup}

We run string regeneration experiments on various
data sets. We use the following data sets:
%
\begin{itemize}
  \item MT08-nw: the first English reference for the newswire portion of the
    Arabic-English translation task for the NIST Open Machine Translation 2008
    Evaluation.\footnote{\url{http://www.itl.nist.gov/iad/mig/tests/mt/2008}}
  \item MT09-nw: the first English reference for the newswire portion of the
    Arabic-English translation task for the NIST Open Machine Translation 2009
    Evaluation.\footnote{\url{http://www.itl.nist.gov/iad/mig/tests/mt/2009}}.
  \item WSJ22: Section 22 from the Penn Treebank.
  \item WSJ23: Section 23 from the Penn Treebank.
\end{itemize}
%
The test sets are first tokenised and lowercased.
Then, $n$-gram rules up to length 5 for each of the processed test sets
are extracted from a large monolingual text collection.
For MT08-nw and MT09-nw, we use all available monolingual data
for the NIST Open Machine Translation 2012
Evaluation.\footnote{\url{http://www.nist.gov/itl/iad/mig/openmt12.cfm}}
This consists of approximately 10.6 billion words.
For WSJ22 and WSJ23, we use all available monolingual data for the WMT13
evaluation, described in \autoref{tab:monolingualStats}.
This consists of approximately 7.8 billion words.
For MT08-nw and MT09-nw, we estimate a modified Kneser-Ney 4-gram language model
on approximately 1.3 billion words of English text, including
the AFP and Xinhua portions of the Gigaword corpus
version 4 and the English side of various Arabic-English
parallel corpora used in MT evaluations. For WSJ22 and WSJ23, we
estimate a modified Kneser-Ney 4-gram language model on all available
monolingual data for the WMT13 evaluation.

\subsection{Baseline}
\label{sec:gyroBaseline}

We first run a baseline on our various test sets with the NgramGen decoder.
Because, there are no constraints on the input length, some long sentences
need more pruning for the decoder to complete the procedure.
Therefore, we use a length dependent histogram pruning of
$\frac{11000}{\text{length(input)}}$. For example, for an input sentence
of length 100, this means that we use a histogram pruning of 110 for that
particular sentence. Results are reported in \autoref{tab:gyroBaseline}.
%
\begin{table}
  \begin{center}
  \begin{tabular}{l|l}
    Test Set & BLEU \\
    \hline
    MT08-nw & 42.32 \\
    MT09-nw & 37.29 \\
    WSJ22 & 44.01 \\
    WSJ23 & 45.48 \\
  \end{tabular}
  \caption{Baseline for the NgramGen Decoder on the various test sets used throughout experimentation. No
    limits are imposed on the input length, thus pruning setting are set so that decoding is feasible on
    the given hardware. Performance is measured with case insensitive BLEU.}
  \label{tab:gyroBaseline}
  \end{center}
\end{table}


We also compare the performance obtained by the NgramGen decoder with previous
work on the data set WSJ23 in \autoref{tab:gyroComparedPreviousWork}. We can
see that our method, without using any additional information apart from the
input bag-of-word, achieves state-of-the-art performance.
All other methods keep base noun phrases as a single unit and also use additional
syntactic information such as part-of-speech tags.
%
\begin{table}
\begin{center}
  \begin{tabular}{l|l|l}
    Work & BLEU & Additional Input Information \\
    \hline
    \citep{wan-dras-dale-paris:2009:EACL} & 33.7 & POS tag + base noun phrases\\
    \citep{zhang-clark:2011:EMNLP} & 40.1 & POS tag + base noun phrases \\
    \citep{zhang-blackwood-clark:2012:EACL2012} & 43.8 & POS tag + base noun phrases \\
    \citep{zhang:2013:IJCAI} & 44.7 & base noun phrases \\
    This work & \textbf{45.5} & \textbf{none}
  \end{tabular}
\end{center}
\caption{Comparing the NgramGen decoder performance with previous work
  on the WSJ23 data set. Case-insensitive BLEU scores are reported.
  The NgramGen decoder achieves state-of-the-art
  performance, without using additional information to the input bag of words.}
\label{tab:gyroComparedPreviousWork}
\end{table}

In the section to follow, we impose length restrictions on
the input bag of words in order to allow more rapid experimentation.

\subsection{Sentence Splitting}
\label{sec:sentenceSplitting}

%TODOFINAL (already done ?) remark about the fact that when input is translation output,
%sentence splitting is not cheating
Running the NgramGen decoder without limiting the length of the input
as in \autoref{sec:gyroBaseline} can be time and memory consuming.
For more rapid experimentation, we only reorder chunks of a limited
size in the
input. The input is divided each time a punctuation sign such
as a comma or a semi-colon is observed or after having seen a maximum
number of tokens.
The implementation simply considers only
$n$-grams whose coverage falls completely into
the chunk being reordered; thus the language model
is computed correctly at the boundaries between chunks.
We also use a histogram pruning of 1000 states
regardless of the input length.

With this restriction, we obtain the results reported
in \autoref{tab:gyroBaselineChopping}. Logically, performance
increases as the maximum size of chunks decreases.
For a maximum chunk length of 1, the BLEU score would be 100.
We will use this baseline
for more rapid experimentation and further analysis.
Specifically, in experiments to follow,
we will conduct experiments with a relatively short maximum chunk
length of 11 and a more realistic maximum chunk length of 20.
%
\begin{table}
  \begin{center}
  \begin{tabular}{l|l}
    Maximum Chunk Length & MT08-nw \\
    \hline
     7 & 82.83 \\
     9 & 78.53 \\
     11 & 74.86 \\
     13 & 71.75 \\
     15 & 68.95 \\
     20 & 64.36 \\
  \end{tabular}
  \caption{NgramGen decoder baseline with restrictions on the input length.
    Case-insensitive BLEU scores are reported. For further experiments, maximum
    chunk lengths of 11 and 20 will be used to contrast results on shorter or longer
    inputs. The BLEU score for a maximum chunk length of 1 would evidently be 100.}
  \label{tab:gyroBaselineChopping}
  \end{center}
\end{table}
%

It should be emphasised that the purpose of splitting the input bag of words by using
information from the sentence to be regenerated is only to be able
to investigate more quickly how to better use the regeneration decoder. This
method should not be used for evaluating an end-to-end system on the string regeneration
task. However, we will see in \autoref{chap:gyroTrans} that using this method
on the output of a translation system is justified since no information from the reference
translation is used.

\subsection{Pruning}
\label{sec:gyroPruning}

In \autoref{sec:sentenceSplitting}, we have shown how to split the input
into chunks to allow more rapid experimentation. Using maximum chunk lengths
of 11 and 20, we will study the effect of various histogram pruning
thresholds.

Results are reported in \autoref{tab:gyroPruning}.
In both maximum input
length settings, performance increases as the maximum number of states per
column increases. For a maximum input length of 11, performance improves by
0.39 BLEU (0.5\% relative, row 1 vs.\ row 7). For a maximum input length of 20, performance improves
by 1.18 BLEU (1.8\% relative, row 8 vs.\ row 14). This is due to a larger search space being
explored in the case of a maximum input length of 20.
%
\begin{table}
  \begin{center}
  \begin{tabular}{l|l|l|l}
    Row & Histogram Pruning & Max Input Length & MT08-nw \\
    \hline
    1 & 500  & 11 & 74.67 \\
    2 & 1000 & 11 & 74.86 \\
    3 & 1500 & 11 & 74.98 \\
    4 & 2000 & 11 & 75.02 \\
    5 & 3000 & 11 & 75.09 \\
    6 & 4000 & 11 & 75.08 \\
    7 & 5000 & 11 & 75.06 \\
%    6000 & 11 & 75.12 \\
%    7000 & 11 & 75.15 \\
%    8000 & 11 & 75.15 \\
%    9000 & 11 & 75.14 \\
%    10000 & 11 & 75.14 \\
    \hline
    8 & 500  & 20 & 64.21 \\
    9 & 1000 & 20 & 64.36 \\
    10 & 1500 & 20 & 64.85 \\
    11 & 2000 & 20 & 64.94 \\
    12 & 3000 & 20 & 65.17 \\
    13 & 4000 & 20 & 65.33 \\
    14 & 5000 & 20 & 65.39 \\
%    6000 & 20 & RUNS \\
%    7000 & 20 & RUNS \\
%    8000 & 20 & RUNS \\
%    9000 & 20 & RUNS \\
%    10000 & 20 & RUNS \\
  \end{tabular}
  \caption{Effect of histogram pruning on performance with a maximum input length
    of 11 and 20. Case-insensitive BLEU scores are reported. 
    In both maximum input length settings,
    increasing the maximum column size increases performance, but
    more so with maximum input length of 20 since the search space
    is much larger in that case.}
  \label{tab:gyroPruning}
  \end{center}
\end{table}

% TODOFINAL make this a graph
% TODOFINAL finish 6000 and above

\subsection{$n$-gram Order for $n$-gram Rules}
\label{sec:gyroEffectNgramOrder}

As mentioned in \autoref{sec:gyroBackground}, previous
work complements the use of $n$-gram language models with
syntactic information for the string regeneration task or for the tree
linearisation task. In this work, $n$-gram language models are complemented
by the use of $n$-gram rules extracted from large amounts of monolingual text.
In this section, we observe the impact of using all rules from unigrams
up to 5-grams or only using rules with specific lengths.
Specifically, we compare the following configurations:
%
\begin{itemize}
  \item All $n$-gram orders: this is the default used in experiments so far.
  \item Unigrams, 4-grams and 5-grams only: our motivation for this configuration
    is to only use rules that are likely to produce a fluent output, namely 4-grams
    and 5-grams. We always use unigrams in order to obtain complete hypotheses.
    For example, if we only used 4-grams and 5-grams, an
    input bag of words of size 6 could
    never be covered, unless overlapping rules were allowed (see \autoref{sec:overlap}).
  \item Unigrams and 5-grams only: our motivation is identical to using
    only unigrams, 4-grams and 5-grams.
  \item Unigrams only: in this setting, only the language model
    informs the decoder. This setting is similar to the one mentioned
    in \autoref{sec:gyroBackground} where all permutations are considered
    and ranked by the language model. Our motivation for this setting is to
    investigate whether $n$-gram rules are at all useful for string regeneration.
\end{itemize}
%
Results
are reported in \autoref{tab:gyroVaryNgramLength}.
We can see that for a relatively short maximum of 11 on the input length,
using more $n$-gram orders is always beneficial.
Because the maximum input length is short, the search space can be
explored more thoroughly and this benefits configurations that generate
larger search spaces.
For a more realistic maximum input length of 20, results are more in line
with our initial motivation for each configuration. Comparing row 6 vs.\ row 5,
we can observe that removing lower order $n$-grams is slightly beneficial.
The same trend is observed when comparing row 7 vs.\ row 5.
Finally, for both maximum input length configurations of 11 and 20, it is always
beneficial to use $n$-gram rules with $n > 1$.
%
\begin{table}
  \begin{center}
  \begin{tabular}{l|l|l|l}
    Row & Rule Configuration & Max Input Length & MT08-nw \\
    \hline
    1 & all rules & 11 & 74.86 \\
    2 & 1g/4g/5g & 11 & 74.85 \\
    3 & 1g/5g & 11 & 74.82 \\
    4 & 1g only & 11 & 74.13 \\
    \hline
    5 & all rules & 20 & 64.36 \\
    6 & 1g/4g/5g & 20 & 64.42 \\
    7 & 1g/5g & 20 & 64.74 \\
    8 & 1g only & 20 & 63.36 \\
  \end{tabular}
  \caption{Effect of selecting $n$-gram rules based on their $n$-gram order.
    Case-insensitive BLEU scores are reported.
    For a short maximum input length, using more orders is always beneficial
    because this increases a tractable search space. For a maximum input length
    of 20, retaining only $n$-gram rules with higher order $n$-gram produces
    better results.}
  \label{tab:gyroVaryNgramLength}
  \end{center}
\end{table}

\subsection{Overlapping $n$-gram Rules}
\label{sec:overlap}

%TODOFINAL (already done ?) cite this:
%Tribble, Alicia, Stephan Vogel, and Alex Waibel. ``Overlapping phrase-level translation rules in an SMT engine.'' Natural Language Processing and Knowledge Engineering, 2003. Proceedings. 2003 International Conference on. IEEE, 2003.

In \autoref{sec:canApply}, we defined the \textsc{CanApply}
operation by requiring that the state coverage and the coverage
of the rule used to extend the state be disjoint.
In this section, we relax
this constraint to allow for a coverage intersection
of a certain maximum overlapping length.
The idea of allowing overlapping rules has been used previously
in phrase-based translation for creating phrase tables with
longer rules~\citep{tribble2003overlapping} % TODOFINAL read this paper more carefully
but not in decoding. % TODOFINAL check this claim

Results are presented in
\autoref{tab:overlap}. Configurations for overlapping rules
include no overlap, a maximum overlap of 1 or a maximum overlap of 2.
As in previous experiments, we contrast the effect of overlapping
rules with a maximum input length of either 11 or 20.
Decoding is run in parallel using Sun Grid Engine~\citep{gentzsch:2001:CCG}.
The total decoding time is measured as the sum of time spent
on each job.
We also study the effect of overlapping rules when
all $n$-gram orders are included (rule configuration ``all'') or
when only unigrams and 5-grams are allowed (rule configuration ``1g/5g'').
We hypothesise that overlapping rules may be more beneficial
when lower $n$-gram orders are missing.
However, we observe that
that allowing for overlapping rules does not improve performance, apart from 
increasing the search space and decoding time in general.
We do not use this decoder capability in future experiments.
% TODOFINAL look at effect when no unigrams are used in terms of coverage

\begin{table}
  \begin{center}
  \begin{tabular}{l|l|l|l|l|l}
    Row & Overlap & Max Input Length &  Rule Configuration & MT08-nw & Total Time (s) \\
    \hline
    1 & no overlap & 11 & all & 74.86 & 20075 \\
    2 & 1          & 11 & all & 74.87 & 19685 \\
    3 & 2          & 11 & all & 74.85 & 18200 \\
    \hline
    4 & no overlap & 11 & 1g/5g & 74.82 & 6774 \\
    5 & 1          & 11 & 1g/5g & 74.81 & 8626 \\
    6 & 2          & 11 & 1g/5g & 74.80 & 8805 \\
    \hline
    7 & no overlap & 20 & all & 64.36 & 57285 \\
    8 & 1          & 20 & all & 64.39 & 89989 \\
    9 & 2          & 20 & all & 64.28 & 95775 \\
    \hline
    10 & no overlap & 20 & 1g/5g & 64.74 & 13420 \\
    11 & 1          & 20 & 1g/5g & 64.72 & 20314 \\
    12 & 2          & 20 & 1g/5g & 64.63 & 22522 \\
  \end{tabular}
  \caption{Effect of allowing for overlapping rules. Case-insensitive BLEU
    scores are reported. In all configurations, allowing for overlapping
    rules does not improve performance. In addition, overlapping
    rules increase decoding time in general.}
  \label{tab:overlap}
  \end{center}
\end{table}

\subsection{Future Cost}
\label{sec:gyroFutureCost}

The justification for adding a future cost estimate to the partial
hypothesis score is analogous to the one provided for phrase-based
translation in \autoref{sec:phraseBasedFutureCost}.
Hypotheses that
cover the same number of words in the input
bag of words are grouped together in a stack and
hypotheses with costs that are too high are pruned out.
However, these hypotheses may not be directly comparable, because
some hypotheses cover very frequent words that are favoured by the
language model while other hypotheses may cover infrequent words.
We therefore estimate a future cost for each hypothesis.
This future cost is added to the partial hypothesis cost for pruning
purposes.

We investigate the effect of using a unigram language model to estimate
the future cost. Given a partial hypothesis, its future cost is defined
by the unigram language model score of the words from the input
bag of words that are not covered by the partial hypothesis.
Because there is no notion of word ordering in the input, we
cannot easily use a higher order $n$-gram language model
language model to estimate future cost.

We compare the use of future cost estimates with maximum input lengths
of 11 and 20, as in previous experimentation. Results are reported in \autoref{tab:futureCost}.
In both settings, adding a unigram language model as
future cost estimate to the cost of a partial hypothesis for pruning is beneficial.
For a short maximum length of 11 on the input length, we
obtain a gain of 0.24 BLEU while for a maximum input length
of 20, we obtain a gain of 0.69 BLEU.
%
\begin{table}
  \begin{center}
    \begin{tabular}{l|l|l}
      Configuration & Max Input Length & MT08-nw \\
      \hline
      no future cost estimate & 11 & 74.86 \\
      future cost estimate & 11 & 75.10 \\
      \hline
      no future cost estimate & 20 & 64.36 \\
      future cost estimate & 20 & 65.05 \\
    \end{tabular}
    \caption{Effect of using a future cost estimate for pruning.
      Case-insensitive BLEU scores are reported.
      Future cost is computed with a unigram language model.
      Future cost is beneficial for both a relatively short maximum input
      length of 11 and for a more realistic maximum input length of 20, with
      respective improvements of 0.24 BLEU and 0.69 BLEU.}
    \label{tab:futureCost}
  \end{center}
\end{table}

\subsection{Rescoring}
\label{gyro:rescoring}

In \autoref{sec:gyroFstBuilding}, we described how the NgramGen decoder
generates an FST that encodes a set of hypotheses.
We choose this output format in order to be able to apply
various lattice rescoring techniques that also utilise this format
as input (see \autoref{sec:background5gRescoring}, \autoref{sec:lmbr}
and \autoref{sec:lmbrSysComb}).

In this section, we show that these rescoring techniques
are not only effective on the output of a first pass translation decoder
but also in our string regeneration setting.
Specifically, we run 5-gram language model lattice rescoring as well
as LMBR lattice rescoring experiments. LMBR applied to hypothesis combination,
described in \autoref{sec:lmbr}, will be applied in \autoref{chap:gyroTrans}.

Results are reported in \autoref{tab:gyroRescoring}, on all our test sets.
The MT08-nw set is used to tune LMBR parameters, and these parameters are
tested on the MT09-nw set. Similarly, WSJ22 is used as a tuning set for WSJ23.
Because we only use a single language model as the only feature and because in all decoder
hypotheses, the length is identical, there is no tuning or testing set
for first-pass decoding or 5-gram language mode rescoring.

We can see that large BLEU gains are obtained in 5-gram rescoring: +1.29 BLEU
on average across all test sets. LMBR rescoring also provides
an average gain of 2.54 BLEU with respect to first pass decoding over the
test sets MT09-nw and WSJ23.
%
\begin{table}
  \begin{center}
  \begin{tabular}{l|l|l|l|l}
    Configuration & MT08-nw & MT09-nw & WSJ22 & WSJ23 \\
    \hline
    first pass & 64.36 & 60.91 & 62.76 & 64.39 \\
    +5g       & 66.11 & 63.26 & 63.18 & 65.04 \\
    +LMBR     & 67.37 & 64.53 & 64.28 & 65.85 \\
  \end{tabular}
  \caption{Effect of lattice rescoring on the first pass lattices obtained
    by the NgramGen decoder. Case-insensitive BLEU scores are reported.
    Rows ``+5g'' and ``+LMBR'' show large BLEU gains with respect to
    first pass decoding, demonstrating the effectiveness of these lattice
    rescoring techniques across various tasks, including translation and
    string regeneration.}
  \label{tab:gyroRescoring}
  \end{center}
\end{table}

\section{Conclusion}

In this chapter, we have presented a novel approach to the string regeneration
task, inspired from phrase-based models for SMT.
We have implemented a decoder, NgramGen, that operates in a similar fashion to phrase-based
stack-based decoders and achieves state-of-the-art
performance on the string regeneration task from a simple bag of word on the Penn Treebank
Section 23 data set.

We have also analysed various capabilities of our decoder.
From this analysis, we draw the following conclusions:
%
\begin{itemize}
  \item Removing lower order $n$-gram rules other than unigrams, which are needed to ensure that the input can be covered entirely, improves performance.
  \item Allowing for overlapping rules slows down the decoding procedure and
    does not provide gains in our experiments.
  \item Future cost estimates improve the pruning procedure and overall performance.
  \item Lattice rescoring techniques are applicable and provide gains in the string
    regeneration setting.
\end{itemize}


In the following chapter, we will explore potential application of our decoder
to word reordering in machine translation. We will consider the output of
a machine translation decoder as the input bag of words for the regeneration decoder.

%TODONEVER dep lm
% TODOFINAL (done ?)      \item analysis of when the input is lost
% TODOFINAL for translation column about translation results + oracle for 1-best or 10-best. show get same score with combination or biased lm.

%There is a separate report on all experiments (attached).
%Here are possible experiments:
%
%\begin{itemize}
%  \item investigate various pruning thresholds and see impact on BLEU. Show oracle scores as well. Maybe compare to the FST-based Gyro software.
%  \item investigate how useful are longer n-gram rules. For example, see the impact of using only unigram rules.
%  \item investigate if it is useful to use overlap in decoding (overlap means that we can cover a input position more than once).
%  \item show gains from rescoring: 5g rescoring, LMBR rescoring and combination with MT hypothesis space.
%\end{itemize}

%\section{String Regeneration Experiments with Dependency LM}

%\begin{itemize}
%  \item Rescoring with dependency LM: gyro generates and n-best list which can be rescored with dependency LMs.
%  \item What happens if we only rescore with left LM, right LM or head LM.
%  \item Compare rescoring with dependency LM directly integrated into Gyro.
%\end{itemize}
