\chapter*{Abstract}

%\vspace*{-1.1cm}

The relatively recently proposed hierarchical phrase-based
translation model for statistical machine translation (SMT)
has achieved state-of-the-art performance
in numerous recent translation evaluations. Hierarchical phrase-based
systems comprise a pipeline of modules with complex interactions.
In this thesis, we propose refinements to the hierarchical
phrase-based model as well as improvements and analyses in various
modules for hierarchical phrase-based systems.

We took the opportunity of increasing amounts of available training
data for machine translation as well as existing frameworks for
distributed computing in order to build better infrastructure
for extraction, estimation and retrieval of hierarchical phrase-based
grammars. We design and implement grammar extraction as a series of
Hadoop MapReduce jobs. We store the resulting grammar using the HFile
format, which
offers competitive trade-offs in terms of efficiency and simplicity.
We demonstrate improvements over two alternative solutions used in
machine translation.

The modular nature of the SMT pipeline, while allowing individual
improvements, has the disadvantage that errors committed by
one module are propagated to the next. This thesis
alleviates this issue between the word alignment module and the
grammar extraction and estimation module by considering richer
statistics from word alignment models in extraction. We use
alignment link and alignment phrase pair posterior probabilities
for grammar extraction and estimation and demonstrate translation
improvements in Chinese to English translation.

This thesis also proposes refinements in grammar and language
modelling both in the context of domain adaptation and in the
context of the interaction between first-pass decoding and
lattice rescoring. We analyse alternative strategies for grammar
and language model cross-domain adaptation. We also
study interactions between first-pass and second-pass language
model in terms of size and $n$-gram order. Finally, we
analyse two smoothing methods for large 5-gram language model
rescoring.

The last two chapters are devoted to the application of
phrase-based grammars to the string regeneration task, which
we consider as a means to study
the fluency of machine translation output.
We design and implement a monolingual phrase-based decoder
for string regeneration and achieve state-of-the-art
performance on this task. By applying our decoder
to the output of a hierarchical phrase-based translation system, we
are able to recover the same level of translation quality as
the translation system.
