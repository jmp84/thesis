\chapter{Summary and Future Work}

% TODONEVER maybe link to specific experiments on blue in each experimental sections

Statistical Machine Translation was originally framed as a source-channel
model with two main components, the translation model and the
language model. This separation into two modules provided opportunities
for improving each module individually, with the hope that improving the language model
would improve fluency while improving the translation model would improve
adequacy. Current state-of-the-art SMT systems include many more components
that interact in complex ways but also provide further individual improvement
opportunities.

In this thesis, we present various improvements and analyses in the
SMT pipeline, with an emphasis on refining hierarchical phrase-based
models. In \autoref{sec:reviewOfWork}, we review the contributions of
this thesis and in \autoref{sec:thesisFutureWork}, we propose
possible future work for each our contributions.

\section{Review of Work}
\label{sec:reviewOfWork}

This thesis demonstrated various refinements in
hierarchical phrase-based translation systems.
A scalable infrastructure was developed for generating
and retrieving from hierarchical grammars. We also
developped an original training method for hierarchical
phrase-based models. Other SMT pipeline components
were refined: we studied the effect of additional
grammar features for domain adaptation and provided recommendations
for language model building in first-pass decoding and rescoring.
Finally, we developped a string regeneration system from the
ground up, analyzed its competitive performance in
the string regeneration task and its potential application to
rescoring machine translation output. We now review each of
these contributions.

\subsection{Hierarchical Phrase-Based Grammars: Infrastructure}

In order to enable more rapid experimentation and to ensure
scalability for hierarchical phrase-based grammar extraction
and retrieval, we developped a flexible MapReduce implementation
of the grammar extraction and estimation
algorithm~\citep{dyer-cordova-mont-lin:2008:WMT}.
Our implementation described in \autoref{chap:hfile} includes
extensions for the computation
of provenance
features~\citep{chiang-deneefe-pust:2011:ACL} and arbitrary
corpus-level features can be added very easily to our framework.

A framework relying on the HFile format
for efficient retrieval of hierarchical rules to
be used in test set decoding was also developped.
Comprehensive time and memory measurements demonstrated
that our system is competitive with respect to linearly
scanning, possibly concurrently, the datastructure containing
the grammar extracted from parallel
text~\citep{pino-waite-byrne:2012:PBML}.

\subsection{Hierarchical Phrase-Based Grammars: Modelling}

In the SMT pipeline, the word alignment module and
the grammar extraction and estimation module usually
communicate through a fixed set of alignment links
produced by the word alignment models. Our contribution
was to allow for more communication between these two
modules~\citep{degispert-pino-byrne:2010:EMNLP}.
In \autoref{chap:extractionFromPosteriors}, we
first reframed the rule extraction procedure
into a general framework. Under this framework, we
introduced two novel methods for grammar extraction
and estimation based on word alignment posterior
probabilities.

The link posterior extraction method was based on alignment
link posterior probabilities and allowed to extract
a greater number of rules more reliably. The phrase pair
posterior method was based on phrase pair alignment posterior
probabilities and was shown to provide a better estimated
translation model. Both methods provided important gains in
translation quality in
a medium size Chinese-English translation task.
Alternative strategies for combining source-to-target
and target-to-source word alignment models were also
investigated; hypothesis combination obtained the best
performance in our experiments.

\subsection{Hierarchical Phrase-Based System Development}

The translation model, refined in \autoref{chap:extractionFromPosteriors},
is one of the main features in the loglinear model for translation.
In \autoref{chap:wmt}, we studied the effect of provenance
features~\citep{chiang-deneefe-pust:2011:ACL} for domain adaptation.
The concept of provenance feature was extended from provenance lexical translation
features to provenance translation features. We demonstrated that,
when rule filtering thresholds are applied for each provenance and all
surviving rules are kept, translation gains can be obtained.

We also provided recommendations for language model modelling both
in first-pass translation and rescoring. In the context of
domain adaptation, we found that the best strategy for language modelling
was off-line linear interpolation of various language models with
weights tuned for perplexity on a held-out development set as opposed
to loglinear interpolation with weights tuned by MERT.
We also observed that when a few billion words of monolingual data were
available, a two-pass strategy of first-pass decoding with a 4-gram
language model trained on all data followed by 5-gram rescoring
provided the best performance in our experiments.
Finally we confirmed in the rescoring setting previous
observations~\citep{brants-popat-xu-och-dean:2007:EMNLP-CoNLL} in
the first-pass decoding setting
that Kneser-Ney smoothing produced better translation quality than
Stupid Backoff smoothing. Thus we obtained translation quality gains
over the top ranking system in the WMT13 Russian-English
task~\citep{pino-waite-xiao-degispert-flego-byrne:2013:WMT}.

\subsection{Fluency in Hierarchical Phrase-Based Systems}

One main reason for the lack of fluency in machine translation
output is a discrepancy in word order between distant language pairs
such as Chinese and English. In \autoref{chap:gyro}, we
studied the problem of word ordering in isolation through
the string regeneration task. We described a regeneration
model inspired from the phrase-based translation model. We also
built a regeneration decoder
analogous to a stack-based decoder for translation.
We achieved state-of-the-art performance on the string regeneration
task. Finally, we analyzed various capabilities of the decoder.

In \autoref{chap:gyroTrans}, we applied string regeneration
to the output of a translation system in a 10-best rescoring setting.
Performance was degraded, however, by combining information obtained
from the translation system with the regeneration system, we were
able to match the quality of the translation output.

\section{Future Work}
\label{sec:thesisFutureWork}

%hfile:
%sentence specific grammar
%parallelize by source sentence
%parallelize by query
%generate source pattern instances more quickly
% real time at the sentence level

The retrieval framework presented in \autoref{chap:hfile}
can exploit parallelization in two ways for further
speed improvements. First, source pattern instances
can be generated for each test set source sentence individually; then
each source sentence can processed in parallel. Because
the HFile format does not multithreaded reads, % TODOFINAL check this on the newest versions
one way to implement this solution is to reuse the multiprocessing
capability of the MapReduce framework.
Another opportunity for parallelization similar to the one just described
is to process queries in parallel. Again, care has to be taken to use
a multiprocessing approach rather than a multhreading approach.
Finally, the source pattern instance creation step can be fine-tuned
in order to approach the speed of a real time system at the sentence
level.

%posteriors:
%source constraints
%alignment constraints
%target constraints
%ranking function
%counting function
%for all these, explore other models than HMM

In \autoref{chap:extractionFromPosteriors}, the rule extraction
algorithm was described in a framework involving
source constraints, alignment constraints, target constraints, a ranking
function and a counting function. This provides an opportunity
for future work by investigating alternatives to the definitions
of these constraints and ranking and counting functions.
Specifically, other word alignment models may be explored.

% system development
% scale up 5-gram language model data

In \autoref{chap:wmt}, we recommended to use a two-pass
decoding strategy and we also concluded that Kneser-Ney
smoothing was beneficial over Stupid Backoff smoothing
in 5-gram language model rescoring. These recommendations
are valid for language models built on a few billion words.
In the future, we will revise them with language models
trained on order of magnitude more monolingual data.

% string regeneration
% investigate the use of overlap and use no unigrams
% investigate more features

We successfully applied phrase-based models and decoding
techniques to the task of string regeneration
in \autoref{chap:gyro}. However, interesting future work may
be carried out to investigate the effect of using more features
than simply a language model. In addition, we may study
how overlapping rules can be used to obtain more chances
of covering the entire input if no unigram rules are used.

The effect of additional features for the regeneration
decoder may also be studied in the regeneration of
translation output setting. If translation quality
as measured BLEU does not increase, efforts will be
dedicated to improve fluency while maintaining translation
quality. Finally, more sophisticated future cost estimates
may be studied, such as $n$-gram language model with $n > 1$: this
is possible because when regenerating translation output, the word
ordering information is legitimately available in the input.

% string regeneration for translation
% fluency
% more features
% TODONEVER mention dependency language model
